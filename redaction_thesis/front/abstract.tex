\section*{Résumé}
\addcontentsline{toc}{chapter}{Résumé}
Ce mémoire de recherche présente une analyse numérique d'une collection diversifiée d'articles scientifiques couvrant l'histoire, la géographie, l'archéologie et les sociétés de la péninsule indochinoise de la fin du XIXe siècle à la seconde moitié du XXe siècle. En employant des techniques de linguistique informatique telles que la modélisation thématique et les méthodologies distributionnelles, cette étude cherche à découvrir l'interaction complexe du langage scientifique transdisciplinaire dans le contexte colonial. Le corpus, composé de textes écrits en plusieurs langues, présente également une  inestimable opportunité d'explorer la culture multiforme de cette région. L’étude vise à mettre en lumière l’importance des sources de données historiques coloniales peu connues et suggère les moyens de les exploiter plus efficacement.


\medskip

\textbf{Mots-clés: société savante; science coloniale ; discours scientifique ; discours colonial ; indochine ; société savante ; vecteurs de mots ; modélisation de sujets ; humanités numériques ; lecture distante }

\textbf{Informations bibliographiques:} Ngoc Giang Nguyen, \textit{Exploration numérique d’un corpus
scientifique colonial: Le cas du Bulletin de la sociétés des études indochinoises (1883-1975)}, mémoire de master 2 \og Humanités Numériques\fg{}, dir. [Pascal Bourdeaux, Marc Bui], Université Paris, Sciences \& Lettres, 2023.



\section*{Abstract}
\addcontentsline{toc}{chapter}{Abstract}
This thesis presents a digital analysis of a diverse collection of scientific papers spanning the history, geography, archaeology, and societies of the Indochinese Peninsula from the late 19th century to the second half of the 20th century. This study seeks to uncover the intricate interplay in the transdisciplinary languages of science within the colonial context by employing computational linguistics techniques like topic modeling and distributional methodologies. The corpus, comprising written texts in several languages, also presents an invaluable opportunity to explore the multifaceted culture of this region. The study highlights the importance of little-known colonial historical data sources and suggests ways to exploit them more effectively.


\medskip

\textbf{Keywords:  société savante ; colonial sciences, scientific discourse ; indochina ; word vectors ; topic modeling ; digital humanities ; distant reading }

\textbf{Bibliographic Information:} Ngoc Giang Nguyen, \textit{Digital exploration of a colonial scientific corpus: The case of the Bulletin de la société des études indochinoises (1883-1975)}, M.A. thesis \og Digital Humanities\fg{}, dir. [Pascal Bourdeaux, Marc Bui], Université Paris, Sciences \& Lettres, 2023.


\clearpage