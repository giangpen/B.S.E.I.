\part*{Introduction}
\addcontentsline{toc}{part}{Introduction}
\markboth{Introduction}{Introduction}

\vspace*{\fill}
\epigraph{Besoin de connaissance, sous le triple aspect de l’apprentissage, de la création, de la vulgarisation ; 
-	Besoin d’intégrer son passage éphémère dans une œuvre qui dure ; 
-	Besoin d’appuyer son effort individuel sur celui des devanciers et des contemporains qu’anime la même fois ;
-	Besoin de trouver réunis, pour un agrandissement réciproque de culture général, des artistes, des littératures, des savants ; 
-	Besoin d’utiliser le meilleur terrain d’entente entre Asiatiques qui se cherchent et Européens qui s’étonnent, unis, l'occasion en en rare pour une commune tache désintéressée.}{\textit{Raphaël Barquissau (Bulletin de la société des études indochinoises 1933)}}


\vfill\clearpage
Vingt-quatre ans après que le gouvernement français a pris le contrôle de Saigon, qui deviendrait ultérieurement la capitale de l'Union indochinoise, un bulletin scientifique est apparu dans cette ville en janvier 1883 dans le cadre de l'Empire colonial français\footcites{paris_2016}. Il poursuit sa mission jusqu'en 1975, lorsque le dernier soldat américain s'est retiré du Vietnam. Pendant cette période s'étendant sur près d'un siècle, soit 92 ans, tous les événements politiques majeurs, les changements de régime de la part des Français, des Japonais, des Américains ou des Vietnamiens, n'auront pas d'influences sur l'organisation des réunions mensuelles d'une société savante ni l'édition trimestrielle de son Bulletin scientifique. Cette pérennité est exemplifiée par l'existence continue de la Société des Études indochinoises et de son périodique, le Bulletin de la Société des Études Indochinoises, durant cette longue période allant de 1883 à 1975.

L'une des réalisations importantes apportées par les Français en Indochine a été la construction et le développement de l'industrie de l'imprimerie. Les relations entre l'Indochine et la France d'alors jusqu'au milieu du XXe siècle, après tous leurs bouleversements politiques, ont laissé dans l'histoire un trésor diversifié et substantiel de documents à la fois peu connus et peu exploités. Actuellement, une partie de ces archives est disponible à "La BnV\footnote{Bibliothèque nationale du Vietnam} et la BnF\footnote{Bibliothèque nationale de France} possèdent les plus importantes collections relatives au Vietnam colonial. C'est pourquoi, parmi les documents présentés sur le site, près de 10 000 d'entre eux sont issus du Dépôt légal indochinois constitué entre 1922 et 1954, numérisés pour l'occasion."\footcites{good34}
 
Au sein de ces archives, il est important de constater que les revues et journaux scientifiques représentent des composantes capitales de qualité, constituant ainsi le socle même de la science moderne dans divers domaines, ce qui les rend particulièrement intrigants dans le cadre de notre projet. En premier lieu, il convient de noter que le Vietnam, en tant que nation, possède une base scientifique relativement récente dans les sciences humaines. En effet, l'adoption systématique de la science occidentale dans ce pays n'a véritablement débuté qu'au cours du XXe siècle. Il est intéressant de souligner qu'au sein de l'École française d'Extrême-Orient (EFEO), les Vietnamiens ont été intégrés au personnel scientifique en qualité d'assistants dès 1929, puis en tant que chercheurs à partir de 1939\footcites{04}. Cette évolution marque une étape significative dans l'histoire de la recherche scientifique dans la région. 

Deuxièmement, il est plausible d'envisager que l'insuffisance de références documentaires ou l'absence de mention des colonies au sein des métropoles, ainsi que la barrière linguistique qui entrave les chercheurs au Laos, au Vietnam ou au Cambodge, d'anciennes colonies où l'usage de la langue française a graduellement décliné après l'obtention de l'indépendance, ont pu contribuer à créer un hiatus entre les sources de recherche disponibles et l'objet même de leur recherche. Cette disparité entre les ressources documentaires et les sujets d'étude peut être source de défis pour les chercheurs.

Ainsi, les archives scientifiques coloniales peuvent être considérées comme une manifestation de l'histoire des sciences en Indochine, et elles pourraient également être interprétées comme une mémoire collective du savoir. À cette époque, la recherche sur l'Union indochinoise était entreprise par des scientifiques et des auteurs issus de diverses disciplines, langues et cultures. Dans le cadre du corpus de la Société des Études Indochinoises, le quoc-ngu (la romanisation du vietnamien), le chinois, le laotien, le cambodgien (khmer) coexistaient avec le français en tant que langues prédominantes des articles.

L'accès libre aux données a ouvert de nouvelles perspectives pour approfondir notre compréhension des ressources dont nous disposons. Il est important de souligner que les recherches scientifiques de qualité sur l'Indochine du XIXe et XXe siècles, menées depuis son territoire, demeurent relativement rares, et la majorité d'entre elles sont rédigées en français. À travers ces sources, nous sommes en mesure d'explorer des connaissances cruciales, telles que les diverses manières dont la science a été pratiquée, utilisée et influencée par le phénomène de colonisation. 
Il s'agit là d'axes de recherche majeurs, englobant la diversité culturelle en Indochine, l'archéologie et l'histoire ancienne. 

De plus, ces documents fournissent des informations essentielles concernant la période de colonisation française en Indochine, les structures sociales qui la caractérisent, ainsi que l'éducation et les systèmes de croyances locaux. 
Ils apportent également des éclairages sur la faune, la flore et les enjeux environnementaux auxquels la région était confrontée à l'époque, ainsi que sur les problématiques liées à l'agriculture et à la médecine. 
En outre, ces écrits comprennent des témoignages et des récits de première main provenant de résidents d'Indochine de l'époque, qu'ils fussent Français, Indochinois ou d'autres nationalités. Bien que ces connaissances semblent être demeurées relativement indépendantes de la vie quotidienne des habitants ordinaires de l'Indochine de cette époque, elles ont néanmoins apporté une contribution significative aux avancées scientifiques de cette région. En effet, de nombreux lecteurs et auteurs issus de l'élite intellectuelle ont joué un rôle prépondérant dans le développement de la science indochinoise.


À partir de ce point, nous pouvons élaborer les hypothèses spécifiques concernant les premières conclusions découlant de ce travail : 

\textit{Comment pouvons-nous exploiter un corpus scientifique imprimé à l’état brut à l'aide de l'humanité numérique ? Quelles sont les principales thématiques abordées dans le Bulletin de la Société des Études Indochinoises ? Comment ces sujets ont-ils évolué au fil du temps ? En utilisant des outils numériques, de quelle manière pourrions-nous repérer des documents scientifiques au sein de ce corpus en se basant sur des mots-clés ?}

En ce qui concerne la méthodologie, le principal objectif de cette étude est d'exploiter Kraken\footnote{\cite{kraken}}, un système de Reconnaissance Optique de Caractères optimisé, afin de créer un modèle multilingue. Ce modèle sera principalement axé sur les rédactions du vietnamien et du français, dans le but d'identifier les documents indochinois datant des XIXe et XXe siècles. Les premières étapes de formation impliqueront l'utilisation des articles issus du Bulletin de la Société des Études Indochinoises.

Ensuite, grâce à l'application des techniques de la "Lecture à distance" (Distant Reading\footnote{\cite{distant}}), une approche quantitative élaborée par Franco Moretti qui nous permet d'explorer en profondeur des corpus de grande envergure tout en réduisant la dépendance à l'égard des ressources humaines, nous serons en mesure d'identifier et de mettre en évidence des sujets scientifiques significatifs évoqués au sein du corpus. De plus, nous chercherons des moyens efficaces pour exploiter ces sources en vue de futures recherches. Ces travaux visent à adopter une perspective "distante" sur les données, mettant ainsi en lumière le Bulletin de la Société des Études Indochinoises avec toutes ses caractéristiques distinctives. Il convient de noter que le Bulletin est la publication scientifique emblématique d'une société savante considérée comme la "Doyenne des sociétés savantes en Indochine" par le Comité des Travaux Historiques et Scientifiques\footnote{\cite{comite}}.

Dans la première section de ce mémoire, nous présenterons en détail le corpus, en exposant son contenu et son contexte, soulignant son importance en tant que précieuse base de données textuelles concernant l'Indochine.

La deuxième section se penchera sur les étapes essentielles liées à la conversion d'images PDF en texte, réalisée grâce à l'utilisation de deux logiciels, à savoir Kraken et eScriptorium (une application web de transcription automatique). Nous aborderons également les aspects relatifs au formatage et au traitement des données, conduisant ainsi à la création d'un corpus accessible par le biais des méthodes de Traitement du Langage Naturel (NLP). En outre, nous dévoilerons les premières investigations effectuées sur ce corpus, offrant ainsi une première approche des connaissances qu'il renferme sur divers sujets.

Nous orienterons notre attention en particulier vers les domaines scientifiques, car ils constituaient le thème prédominant des articles publiés dans le Bulletin à l'époque. Notre objectif principal est d'identifier les principaux domaines de recherche traités au fil de l'existence de cette revue. Pour ce faire, nous appliquerons un modèle bien établi en matière de Modélisation de Sujets, à savoir Top2Vec. Nous entreprendrons un processus de formation en optant pour une configuration bien adaptée au modèle, ce qui nous permettra d'obtenir une première structure de regroupement des documents.

Nous apporterons certaines adaptations dans la phase de préparation des données afin d'améliorer la qualité des regroupements. De plus, nous tirerons parti de interfaces de programmation (API) proposées par Top2Vec pour optimiser notre apprentissage automatique.

Nous exposerons une analyse humaine approfondie afin de garantir une compréhension précise de l'exactitude de chaque groupe de sujets identifiés. Tout en mettant en lumière les limites de cette méthode d'entraînement, notamment son inefficacité à découvrir des thèmes enfouis au sein de sujets généraux, nous proposons l'intégration d'un modèle Word2Vec dans notre processus de modélisation. Cette intégration pourrait renforcer la sélection de documents en raison de sa capacité à saisir et à représenter de manière significative les relations sémantiques entre les mots.

Le mémoire est structuré en trois chapitres. Il commence par une étude préliminaire et un examen de l'état de l'art, explorant le cadre théorique qui traverse l'histoire des sciences de l'Occident à l'Orient. Le deuxième chapitre se penche sur la mise en pratique des humanités numériques, suivi des résultats obtenus grâce à une exploration générale du corpus. Enfin, dans le troisième chapitre, nous abordons le problème de la modélisation des sujets en utilisant différentes approches techniques. Le mémoire se conclut par une synthèse des résultats et des perspectives pour de futures recherches.