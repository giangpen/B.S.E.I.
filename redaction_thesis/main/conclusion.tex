\part*{Conclusion}
\addcontentsline{toc}{part}{Conclusion}
\markboth{Conclusion}{Conclusion} 

L'étude du le Bulletin de la Société des Études Indochinoises (B.S.E.I) n'est pas seulement l'étude d'une revue scientifique mais aussi l'étude d'une organisation scientifique avec tous les appareils qui la sous-tendent, une coulisse scientifique affichée sur papier. Tous les procès-verbaux et activités de l'association à l'intérieur et à l'extérieur du journal créent un cadre de vie scientifique vivant en Indochine. Un sérieux et une passion pour la recherche avec un attachement toujours plus profond à cette terre. D'une autre manière, on peut conclure que grâce à la revue, la société savante a favorisé l'émergence d'une communauté scientifique indépendante, dynamique et particulièrement productive en Indochine, atteignant un niveau de qualité sans précédent tant avant qu'après cette période.

La science en Indochine, même en adoptant des concepts européens, ne pouvait pas produire des résultats identiques à ceux de l'Europe. La science coloniale, en général, était influencée par les modèles européens, mais elle devait nécessairement s'adapter aux réalités locales, notamment dans un pays caractérisé par une culture asiatique fortement influencée par la pensée classique de la Chine et de l'Inde.

La société savante S.E.I. et sa revue s'inscrivent effectivement dans le concept de société savante, avec toutes les caractéristiques typiques de ce type d'organisation, mais aussi avec des exceptions dues au contexte colonial particulier de l'Indochine. Les scientifiques qui travaillaient en Indochine étaient véritablement des explorateurs, car l'environnement indochinois était un lieu propice pour tester leurs méthodes de raisonnement européennes tout en découvrant des méthodes et des approches qui n'avaient jamais existé auparavant. En tant que pionniers dans cette terre scientifique nouvelle, ils n'avaient pas d'empreintes de prédécesseurs à suivre, ce qui leur offrait une grande liberté d'exploration intellectuelle.

Dans les premières années du B.S.E.I., même les futurs "spécialistes" n'avaient pas toujours une compréhension parfaite de leur domaine, n'étaient pas sûrs de leurs connaissances initiales et ne disposaient pas nécessairement de résultats clairs dans leurs recherches. Cependant, au fil du temps, les sociétaires ont pu affiner leurs méthodes de recherche, accumuler des connaissances et contribuer de manière de plus en plus significative à la compréhension de l'Indochine et de ses spécificités scientifiques. Naturellement, ils se sont transformés en érudits de l'Indochine, des spécialistes du Vietnam, des individus dont la compréhension de cette région coloniale était exceptionnelle et prééminente

Après toutes les premières découvertes de cette recherche dans le cadre d'un programme de master 2, la première porte vient de s'ouvrir et les mystères de la plus ancienne revue scientifique d'Indochine perdurent encore. Nous espérons poursuivre ces recherches plus en profondeur et obtenir des résultats positifs par la suite.

Il reste encore de nombreuses orientations pour l'avenir de ce projet, des aspects inexplorés dans les documents d'organisation de l'association, les listes de membres qui ont changé au fil des périodes, la capacité à organiser les finances, la période de dépassement des crises et des difficultés. Toutes ces connaissances restent ouvertes à des recherches plus approfondies sur ce projet.

Sur le plan technique, la première réalisation du projet réside dans la mise en place d'un modèle de reconnaissance optique de caractères (OCR) pour l'écriture manuscrite, qui atteint un taux de réussite à l'entraînement relativement élevé de 93,2\%. De plus, ce modèle démontre une capacité à se perfectionner davantage en utilisant de nouvelles données et d'autres méthodes d'entraînement. 

Les limites et les échecs rencontrés lors de la construction d'un modèle OCR pour le vietnamien ouvrent également de nouvelles opportunités pour des expérimentations futures. Cela comprend la possibilité de concevoir des fichiers ALTO adaptables spécifiquement adaptés à ce corpus, ainsi que l'exploration de solutions avec d'autres logiciels mieux adaptés au vietnamien. Malgré les défis persistants dans la reconnaissance des textes mis en évidence par les expérimentations réelles avec ce corpus, ce modèle a tout de même réussi à créer une base de textes compréhensible. Cela ouvre la voie à une exploration plus approfondie du corpus et, potentiellement, à une modélisation significative des sujets qu'il renferme.

De plus, l'application des modèles de Topic Modeling s'avère extrêmement efficace pour identifier les sujets principaux présents dans le corpus. Bien que le paramétrage des modèles d'apprentissage puisse être délicat en raison de la complexité et de la diversité des documents, nous avons néanmoins développé une méthode de recherche en combinaison des modèles Top2Vec et Word2Vec qui nous permet finalement d'accéder aussi précisément que possible aux thèmes scientifiques que nous cherchons à étudier. Ces premiers résultats nous encouragent également à envisager la création d'un moteur de recherche thématique plus automatisé et plus efficace pour notre base de données. Ils constituent aussi une base importante pour la poursuite de nos recherches dans ce riche réservoir de documents historiques et de connaissances sur l'Indochine.
